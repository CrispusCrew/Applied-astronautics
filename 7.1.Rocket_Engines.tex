\documentclass{article}

% Language setting
% Replace `english' with e.g. `spanish' to change the document language
\usepackage[russian]{babel}

% Set page size and margins
\usepackage[a4paper,top=2cm,bottom=2cm,left=2cm,right=2cm,marginparwidth=1.75cm]{geometry}

% Useful packages
\usepackage{amsmath}
\usepackage{graphicx}
\usepackage[colorlinks=true, allcolors=blue]{hyperref}
%\usepackage{indentfirst}
\setlength{\parskip}{1em}
\title{Модуль "Прикладная космонавтика" \\  7.1. Ракетные двигатели}
\large
\author{Габзетдинов Р.И.\\ Университетская гимназия} 
\date{}
\s
\begin{document}
\maketitle
\setlength{\parindent}{0}
\noindent{\textit{\small{Если в этой, или других методичках и материалах вы найдете ошибку или опечатку, просьба написать об этом \quad t.me/Samnfuter \quad vk.com/gabzetdinoff \quad crispuscrew71@gmail.com}}}
\large
\section{Общие понятия о ракетном двигателе}
\textbf{Ракетный двигатель (Rocket engine)} - частный пример  \textbf{реактивного двигателя}, т.е. двигателя создающего тягу за счет выброса рабочего тела в противоположенном направлении от желаемого направления тяги. Отличается от прочих реактивных использованием только того рабочего тела, что  \textbf{запасенно на аппарате}, в отличие, от например, воздушно-реактивных двигателей, которые используют кислород из атмосферы как окислитель.

\noindent{\small{Далее будут использованны сокращения:
РД/RE - ракетный двигатель/rocket engine, ЖРД/LRE/LR - жидкостный РД/liquid-propellant rocket/liquid rocket, ТТРД/SRB - твердотопливный РД/solid rocket booster/solid rocket, ЭРД - электрический РД/electric propulsion, ЯРД/NTR - ядерный РД./nuclear propulsion/nuclear thermal rocket - NTR наиболее распостраненный вид ЯРД, и почти всегда речь идет про него, если иначе то это будет упомянуто.}}

\section{Характеристики ракетных двигателей}
\begin{enumerate}
    \item $\textbf{Тяга (Thrust) / F}$ \\ \\ Основная задача ракетного, как и любого другого двигателя – это заставить аппарат двигаться, и как следствие, основополагающей характеристикой РД является именно та сила с которой он может “толкать” корабль. Размерность \textbf{Ньютон, килограмм-сила (тонн-сила), фунт(LB)}. Наиболее маломощные двигатели имеют тягу порядка миллиньютонов. Наиболее мощные же – F-1, РД-171М, Space Shuttle SRB, порядков меганьютонов и десятков меганьютонов. \\ 
    \item \textbf{Удельный импульс (Specific impulse) / УИ / Удельная тяга / I / $\textbf I_{sp}$ } \\ \\ \textit{\textbf{Формула Циолкосвкого} - $\Delta V = I_{sp} \cdot ln(\cfrac{M_{wet}}{M_{dry}})$} \\ \\\small{$\Delta V$-характеристическая скорость\quad $I_{sp}$-удельный импульс \quad $M_{wet}$-полная масса\quad $M_{dry$-сухая масса}
    \\ \large
    \\ \textbf{Эффективность} для РД оссобенно критична, т.к. по формуле Циалковского, характеристическая скорость зависит от соотношения масс логарифмически, а \textbf{от УИ - линейно}.  Размерность \textbf{секунды, метр в секунду}. Для двигателей, реакционная масса которых полностью совпадает с рабочим телом, Удельный импульс [секунды] прямо пропорцианален (через стандартное ускорение свободного падения g) Удельной тяге[метры в секунду], т.е. \textbf{[секунды] $\cdot$ g = [метры в секунду]}  Типичные показатели: 100-500 секунд ЖРД, 100-300 секунд ТТРД, 700-1200 секунд ЯРД, $\sim 10^3$ и $10^4$ секунд для ЭРД и других высокоэффективных РД. \\
    \item \textbf{Топливная пара/Монотопливо/Рабочее тело/Rocket propellant} \\ \\
    Выбор топливной пары является основополагающим при проектировке РД, в частности, он \textbf{влияет на криогенность топлива} (температуру кипения), \textbf{особые условия хранения и заправки)}(взрывоопастность, токсичность и т.п.), \textbf{максимальный удельный импульс}, \textbf{объемы и массу баков}, \textbf{стоимость и доступность топлива}, \textbf{сложность/наличие систем зажигания двигателя}.
    При проектировке важно учитывать что: \begin{itemize}
        \item Криогенные компоненты топлива испаряются еще до старта, а также в полете. Поэтому не рекомендуется использовать их для \textbf{длительных миссий}, а так же желательно производить заправку прямо перед стартом, что \textbf{не позволяет экстренно запустить ракету}.
        \item Взрывоопасные и токсичные виды топлива, особенно \textbf{самовоспламеняющиеся при контакте друг с другом компоненты}, требуют особой надежности конструкции и сбор большего количества лицензий и справок, а зачастую вовсе недоступны.
        \item Зачастую для РД требуются особые сорта, казалось бы распространенных видов топлива, например, т.к. ЖРД - сложные и нежные устройства для них требуется \textbf{особый ракетный керосин}, который значительно дороже и сложнее в производстве, конкретнее марки \textbf{RP-1 (США)} и \textbf{ТС-1 (СССР и Россия)}.
        \item Материалы, из которых сделаны стенки баков или конструкция двигателя, могут \textbf{неблагоприятно реагировать с топливом}. Например, химически активные окислители, такие как кислород (особенно жидкий), фтор и амилы могут окислять элементы РД и баков.
        \item Некоторые виды топлива \textbf{(особенно криогенные)} в процессе работы РД или же при простом хранении могут быть \textbf{в состояниях близких к фазовым переходам}, которые крайне нежелательны, пока топливо не выйдет из сопла РД. Ярким примером этого может служить авария произошедшая с Falcon 9 FT 1 сентября 2016 - из-за недоработки при проектировки верссии Full Trust с переохлажденным кислородом не были учтены фазовые переходы вытеснительного газа, что и привело к взрыву.
    \end{itemize}
    Характеристиками самих топливых пар можно считать \textbf{максимальный УИ, плотность, криогенность, химическую агрессивность, токсичность, самовоспламеняемость}. \\
    \item \textbf{Дросселирование / throttling, отклоняемый вектор тяги (ОВТ) / thrust vectoring} \\ \\
    Поддержка ориентации аппарата возможна благодаря различным системам, применение которых не всегда реализуемо, особенно на первых ступенях крупных ракет. В такой ситуации ориентирование ракеты обеспечивается за счет отклоняемого вектора тяги РД. Реализуется ОВТ за счет \textbf{отклоняемого сопла или целого двигателя}. Реактивная струя отклоняется вместе с соплом или двигателем и по ЗСИ создает тягу в нужную сторону.
    \\ \\
    Дросселирование на практике является ничем иным как способностью двигателя \textbf{изменять свою тягу в процентах от максимальной}. Чаще всего реализуется уменьшением количества подаваемого топлива или изменением геометрии частей РД. Может быть крайне полезным для малых маневров, а так же для уменьшения максимальных перегрузок, в частности в конце работы ступени. Многие двигатели, особенно эпохи начала космонавтики, а так же первых ступеней вообще не умеют дросселироваться (unthrottled)
    \\ 
    \item \textbf{Масса и тяговооруженность(thrust-to-weight ratio)/ТВР(TWR)} \\ \\
    Как уже упоминалось ранее, характеристическая скорость зависит не просто от масы аппарата, а от соотношения сухой и полной масс. Именно поэтому масса двигателя и \textbf{соотношение его тяги к силе тяжести на него действующей на поверхности Земли (тяговооруженность)}, которые напрямую влияют на сухую массу достаточно важные показатели. На сегодня, наиболее высокие показатели тяговооруженности имеют двигатели семейств РД-253 и Merlin (более 150). Наиболее высокую массу имеют крупные твердотопливные двигатели (порядка сотен тонн), где само топливо является РД, и наиболее мощные ЖРД, такие как РД-170, F-1 и т.д. (до 10 тонн). 
    \\
    \item \textbf{Количество включений и способы запуска} \\ \\
    Эта характеристика наиболее применима к ЖРД, т.к. \textbf{ТТРД вообще могут включатся только один раз}, а количество запусков \textbf{ЭРД измеряется, по большей части, только изнашиванием компонентов}. Но так или иначе, если вам необходимо совершить несколько маневров, то следует учесть количество зажиганий у вашего двигателя, т.к. оно может быть весьма ограничено. Основными же типами воспламенителей выступают:
    \begin{itemize}
        \item \textbf{Гиперголическое топливо} - самый простой путь - использовать самовоспламеняющиеся при контакте с друг другом компоненты топлива, многоразовость и простота прилагаются.
        \item \textbf{Электрические воспламенители} - создающие электрическую дугу или искровые разряды. Такой тип воспламенителей - одноразовый.
        \item \textbf{Пиротехнические и гиперголические воспламенители} - начинают процесс горения за счет подрыва пиропатрона или же включения в состав топлива процента гиперголических компонентов. Возможно создание двигателей с множеством включений.
        \item \textbf{Пирофорные воспламенители} - вещества способные самовоспламеняться на воздухе называются пирофорами. Такие вещества используются при создании пирофорных воспламенителей - многоразовых и простых. Наиболее известен пирофор в космонатике - триэтилборан.
    \end{itemize}
    \\ \\
    \item \textbf{Цена и аварийность} \\ \\
    Т.к. космонавтика - очень недешёвая вещь, важно как можно сильнее удешевлять любые запуски. Поэтому снижение стоимости двигателя - важнейшая задача для ОКБ его разрабатывающего. Цена за один двигатель может достигать сотни миллионов долларов (RS-25E для SLS). \\ \\
    Аварийность также немаловажная характеристика, а для пилотируемых полетов, возможно, и более. Например двигатели РД-107/108, стоящие на семействе РН Союз и Союз 2, обладают весьма скромными характеристиками, но благодаря невероятной надежности, вызванной в т.ч. и невероятно длительным сроком их активного использования (с 57 года) они до сих пор активно производятся и летают.
\end{enumerate}

\section{Основные виды ракетных двигателей}
\begin{itemize}
    \item \textbf{Твердотопливные РД/Solid Rocket Booster/Solid rocket - ТТРД/SRB} - самый простой, надежный и дешевый из всех видов РД. По сути представляет собой  \textbf{смесь твердого топлива, способного к самоподдерживающейся экзотермической реакции, в результате которой образуется большое количество высокотемпературного газа с большой скоростью истекающего из ТТРД}. \\ \\
    Основными преимуществами данного типа двигателей выступают  \textbf{низкая цена},  \textbf{легкость и скорость производства},  \textbf{отказоустойчивость} за счет своей простоты,  \textbf{возможность создания ТТРД очень высокой тяги без особых сложностей}. \\ \\
    Недостатками же ТТРД являются сложность реализации  \textbf{дросселирования},  \textbf{многократного запуска}, опасность  \textbf{блокирования сопла твердым топливом и продуктами его реакций}, относительно  \textbf{низкий удельный импульс} (наиболее высокий достигнутый на пратике это 285.6 секунд на Titan IVB SRMU). \\ \\
    \item  \textbf{Жидкостный РД/liquid-propellant rocket/liquid rocket - ЖРД/LRE/LR} - самый частоиспользуемый в "большой" космонавтике тип двигателей. Имеет множество подтипов, которые, в свою очередь, тоже имеют подтипы. Зачастую в этот вид относят и газовые РД, т.к. последние редко используются в связи с отсутствием достоинств в сравнении с обычными ЖРД. Самый распространенный тип ЖРД - двухкомпонентные, т.е. чье топливо состоит из двух различных жидких компонентов. \\ \\
    Преимущества ЖРД - относительно \textbf{высокая тяга}(до 8 МН) в сочетании с \textbf{хорошим удельным импульсом} (до 470 с), возможность \textbf{широкого дросселирования} и \textbf{множества включений}, второй по исследованости тип после ТТРД,
\end{itemize}

\section{Подробнее про ЖРД}

\subsection{Особые характеристики ЖРД}

\subsection{Виды топлива}

\subsection{Системы охлаждения}

\subsection{Краткая историческая сводка}


\section{Подробнее про ТТРД}
Общие моменты
\subsection{Виды топлива}

\section{Подробнее про ЭРД}

\subsection{Классические ионные РД/Electrostatic}

\subsection{Плазменые РД/Electrothermal}

\subsection{Магнитоплазменные РД/Electromagnetic}

\subsection{Прочие ЭРД}


\section{Подробнее про ЯРД}

\subsection{Тепловые ядерные двигатели/Nuclear thermal rocket}

\subsection{Прочие виды ЯРД}


\section{Дополнительная полезная информация}
Вернеры
Двигатели холодного газа

\end{document}
