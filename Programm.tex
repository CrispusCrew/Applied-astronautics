\documentclass{article}

% Language setting
% Replace `english' with e.g. `spanish' to change the document language
\usepackage[russian]{babel}

% Set page size and margins

\usepackage[a4paper,top=2cm,bottom=2cm,left=3cm,right=3cm,marginparwidth=1.75cm]{geometry}

% Useful packages
\usepackage{amsmath}
\usepackage{graphicx}
\usepackage[colorlinks=true, allcolors=blue]{hyperref}
\usepackage{indentfirst}
\setlength{\parskip}{1em}
\title{Модуль "Прикладная космонавтика"}
\large
\author{Габзетдинов Р.И.\\ Университетская гимназия} \date{}
\s
\begin{document}
\maketitle
\large
\setlength{\parindent}{0}
\section{Актуальность и адресность модуля}

Знания и умения, полученные на данном модуле, могут быть использованы для проектировки, производства и эксплутуации ракет, ракетно-космических комплексов, спутников и других ПН в ходе разработки и реализации различных проектных работ, в частности НИР, ОКР и НИОКР.

Модуль расчитан на школьников 9-11 класса технически-ориентированных профилей, в первую очередь инженерного.

\section{Цель}

Целью данного модуля, помимо популяризации космической отрасли среди перспективной молодежи, является развитие у учеников навыков решения базовых задач НИР, ОКР и НИОКР, связанных с космической отраслью, теоретических знаний аспектов ракето и спутникостроения, а так же представления о положении дел в космической отрасли.

\section{Задачи}
\begin{enumerate}
    \itemТеоретический минимум астрономия (небесная механика, тела солнечной системы)
    \itemТеоретический минимум системы РН (РД, авионика, стабилизация и ascent path)
    \footnote{РД - ракетные двигатели}
    \itemТеоретический минимум системы платформы КА (ДУ, Ориентация и СУ, СЭП, СС)
    \footnote{ДУ - двигательная установка, СУ - системы управления, СЭП - система электропитания, СС - системы связи}
    \itemТеоретический минимум орбитальное маневрирование (небесная механика, расчет оптимального маневра, гравитационные маневры)
    \itemТеоретический минимум пилотируемые КА (СЖО, скрубберы, САС, СВИП, СРЗ)
    \footnote{СЖО - системы жизнеобеспечения, САС- система аварийного спасения, СВИП - системы возвращения и посадки, СРЗ - системы радиационной защиты}
    \itemКраткая история космонавтики
    \itemИзучение современной космонавтики
    \itemПрактическая работа 1 - расчет маневров АМС "Вояджер - 2"
    \footnote{АМС - автоматическая межпланетная станция}
    \itemПрактическая работа 2 - программа расчета запаса характеритистической скорости для создания ретрансляционной сети n аппаратов
    \itemПрактическая работа 3 - программа выхода РН на НОО, на примере языка KOS
    \footnote{KOS - Kerbal Operating System}
    \itemПрактическая работа 4 - разработка скрипта посадки АМС на Луну на примере языка KOS.
    \itemПрактическая работа 5 - разработка концепта собственного наноспутника формата CubeSat, обосноснование ценности, экономическое исследование
    \itemПрактическая работа 6 - разработка концепта собственного дизайна малого АМС, расчет маневров, обоснование научной ценности, экономическое исследование.
\end{enumerate}

\section{Структура курса}
\begin{tabular}{ | c | l | c | l | }
\hline
№ & Название & Академ. часов & Комментарий  \\ \hline
1 & Вводное занятие & 4 & Первые две недели \\ \hline
2 & Теормин астрономия & 12 & Небмех(8), Солнечная система(4)  \\ \hline
3 & Теормин орб. маневры & 10 & Виды орбит(2), Маневрирование(4),   \\
& & & Межпланетные перелеты(4) \\ \hline
4 & Практическая работа 1 & 4 & Расчет маневров АМС "Вояджер - 2" \\ \hline
5 & Программирование & 8 & Python/C++ KOS \\ \hline
6 & Практическая работа 2 & 4 & Ретрансляторная сеть n аппаратов \\ \hline
7 & Теормин системы РН & 8 & РД(4), ascent path(2), остальное(2) \\  \hline
8 & Практическая работа 3 & 4 & Циклограмма выхода на орбиту. KOS \\ \hline
9 & Теормин системы КА & 6 & Ориентация и СУ(2), СЭП(2), остальное(2) \\ \hline
10 & Практическая работа 4 & 4 & Посадка АМС на Луну. KOS \\ \hline
11 & Теормин & 4 & СЖО, скрубберы, СРЗ(2), САС, СВИП(2) \\
& пилотирумые КА & &  \\ \hline
12 & История космонавтики & 10 & До 65 года(2), лунная гонка(2),  \\
& & & шаттлы и салюты(2), современность(4) \\ \hline
13 & Практическая работа 5 & 6 & Собственный Кубсат \\ \hline
14 & Практическая работа 6 & 8 & Собственная АМС \\ \hline
15 & Дополнительные часы & 12 & Мат. аппарат и повторение \\ \hline 
\hline & Итого & 104 & 2 занятия в неделю - 26 недель (план - 35)\\  \hline
\end{tabular}

\section{Зачет по модулю}
Для зачета по модулю будет достаточно сдать 2 практические работы # 1-4, либо одну из практических работ 5-6
\end{document}